% %=======================================================================
% % Lista de Símbolos (opcional).
% %
% % Deve ser passado o maior (mais largo) dos símbolos utilizados.
% %=======================================================================

% \begin{listadesimbolos}{xxxxxx}
% % \item[$\star$] bloco simétrico
% % \item[$\forall$] para todo
% % \item[$\in$] pertence a
% % \item[$\subset$] subconjunto de 
% % \item[$:$] tal que
% % \item[$\mapsto$] mapeia para
% % \item[$\rightarrow$] tende para
% % \item[$\Rightarrow$] implica que 
% % \item[$\Leftrightarrow$] equivalente a
% % \item[$\exists$] existe
% % \item[$\dot{x}$] derivada temporal de $x$
% % \item[$\|x\|$] norma Euclideana do vetor $x$
% % \item[$I$] matrix identidade de ordem apropriada
% % \item[$I_n$] matrix identidade de ordem $n$
% % \item[$A^T$] matrix transposta da matriz real $A$
% % \item[$Co\{\cdot\}$] invólucro convexo
% % \item[$sat\{\cdot\}$] função saturação
% % \item[$rank\{\cdot\}$] posto de $(\cdot)$
% % \item[$\Re$] conjunto dos números reais
% % \item[$\Re^n$] conjunto dos vetores reais de dimensão $n$
% % \item[$\Re^{n\times n}$] conjunto das matrizes reais de dimensão $n\times n$
% % \item[$\mathbb{Z}$] conjunto dos números inteiros
% % \item[$\blacksquare$] designação para o fim de teorema, corolário e lema
% % \item[$\square$] designação para o fim de prova
% \item[$\bigtriangledown$] designação para o fim de exemplo numérico
% \end{listadesimbolos}

% %=======================================================================
% % Lista de Símbolos (opcional).
% %=======================================================================