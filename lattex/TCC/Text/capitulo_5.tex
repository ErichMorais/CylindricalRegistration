\chapter{Considerações finais}

Neste trabalho foi proposta uma etapa de um sistema computacional responsável por inspecionar rótulos de envazados com formato essencialmente cilíndricos. Aqui, se tratou do problema a partir do ponto de vista do processamento de imagens, com o propósito de corrigir as distorções causadas pelo formato do objeto quando reproduzido no plano do sensor da câmera, bem como a formação de uma panorâmica que abrange todo o perímetro do objeto.

Como o sistema proposto utiliza somente uma câmera física, emulando as outras três com translações do objeto (com passos angulares de 90$^\circ$) a fim de capturar a informação de todo o perímetro da garrafa, assume-se que o ponto no espaço ocupado pela garrafa é o mesmo nas quatro fotografias consideradas na composição da panorâmica. 

As distorções provenientes ao formato do objeto (cilíndrico) são mapeadas através das equações \eqref{equacao:dist_horz} e \eqref{equacao:dist_vert} a partir da variação angular do sistema de coordenadas da imagem. O sistema negligencia distorções provenientes da lente da câmera, sendo esse um fator que pode ser implementado em trabalhos futuros. A calibração desse parâmetro permite um dimensionamento melhor da representação angular real de cada uma das fotografias, auxiliando na montagem do mosaico. Também é importante destacar que o mapeamento proposto por \cite{Lin:2013} exige uma confiança mecânica do sistema de aquisição que, apesar de ser minucioso, não é um grande problema para aplicações industriais.

As compensações geométricas do sistema são realizadas através da seleção de quatro coordenadas (formando um paralelogramo) provenientes ao mapeamento, sendo estes submetidos a uma transformação geométrica projetiva, mudando seu formato para um retângulo. A realização desta operação com todos os pontos do mapa forma uma nova imagem planificada do rótulo.

Por fim, as quatro imagens retificadas são submetidas ao processo de montagem do mosaico. Este, apesar de depender de um sistema mecânico robusto e confiável, se demonstrou minimamente eficiente a ponto de permitir um ponto ideal de costura aceitável para uma inspeção menos delicada.

Trabalhos futuros  também podem ser realizados na direção de avaliar uma figura de mérito adequada para a  comparação de diferentes panorâmicas. Resultados preliminares indicam que as altas frequências das panorâmicas obtidas podem ser suficientes para a avaliação do correto (ou não) posicionamento do rótulo da garrafa, através da comparação com um padrão.

%para incluir a delimitação automática do rótulo analisado através de uma rede neural e/ou segmentação por formas, por exemplo. Com uma calibração da câmera é possível encontrar o ponto máximo de visão alcançado, dimensionando da representação angular de cada uma das imagens com maior precisão. Além disso, há espaços para melhoria no desempenho do sistema; visando a utilização em de tempo real, poderia ser empregado no algoritmo ferramentas de computação paralela/concorrente a fim de propor um processamento mais rápido das imagens.
