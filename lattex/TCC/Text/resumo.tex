%=======================================================================
% Resumo em Português. 
%O resumo deve ressaltar o objetivo, o método, os resultados e as conclusões do documento. A ordem e a extensão destes itens dependem do tipo de resumo (informativo ou indicativo) e do tratamento que cada item recebe no documento original. O resumo deve ser precedido da referência do documento, com exceção do resumo inserido no próprio documento.
% A recomendação é para 150 a 500 palavras.
%=======================================================================

\begin{abstract}
\noindent 
Na indústria de envase de bebidas é comum a demanda por sistemas de visão computacional chamados de inspetores de rótulos. Esses sistemas verificam o posicionamento dos rótulos afixados em garrafas, permitindo a retirada automática das garrafas que estejam fora dos padrões de qualidade. Entretanto, os inspetores comerciais  são importados, possuem alto custo, e, aparentemente, se baseiam em reconhecimento de padrões, requerendo novo treinamento para cada novo modelo de garrafa ou rótulo. Uma possibilidade de solução para verificação do posicionamento dos rótulos é uma abordagem baseada no registro de imagens e na correlação entre imagens. Para tanto, como etapa de pré-processamento, pode-se construir uma imagem panorâmica da superfície das garrafas. Visto isso, neste trabalho é proposto o estudo de uma solução para geração de imagens panorâmicas da superfície de objetos cilíndricos.
\end{abstract}

%=======================================================================
% Resumo em língua estrangeira (obrigatório somente para teses e
% dissertações).
%
% O idioma usado aqui deve necessariamente aparecer nos parâmetros do
% \documentclass, no início do documento.
%=======================================================================
\begin{otherlanguage}{english}
\begin{abstract}
\noindent 
In the beverage filling industry, demand for computer vision systems called label inspectors is common. These systems check the positioning of the labels affixed to the bottles, allowing the automatic removal of bottles that are outside the quality standards. However, commercial inspectors are imported, expensive and seemingly based on pattern recognition, requiring new training for each new bottle or label model. A possible solution to check the positioning of the labels is an approach based on the registration of images and the correlation between images. For this, as a pre-processing step, a panoramic image of the bottles' surface can be constructed. In view of this, this work proposes the study of a solution to generate panoramic images of the surface of cylindrical objects.
\end{abstract}
\end{otherlanguage}
